\documentclass[12pt, reqno]{amsart}

% Standard AMS packages
\usepackage{amsmath}
\usepackage{amsthm}
\usepackage{amssymb}
\usepackage{amsfonts}

% Optional packages for better formatting and utilities
\usepackage{geometry}
\usepackage{graphicx}
\usepackage{hyperref}
\usepackage{enumitem}

% Geometry settings
\geometry{
    letterpaper,
    left=1in,
    right=1in,
    top=1in,
    bottom=1in
}

% Theorem environments
\theoremstyle{plain}
\newtheorem{theorem}{Theorem}[section]
\newtheorem{lemma}[theorem]{Lemma}
\newtheorem{proposition}[theorem]{Proposition}
\newtheorem{corollary}[theorem]{Corollary}

\theoremstyle{definition}
\newtheorem{definition}[theorem]{Definition}
\newtheorem{example}[theorem]{Example}
\newtheorem{exercise}[theorem]{Exercise}

\theoremstyle{remark}
\newtheorem{remark}[theorem]{Remark}

% document metadata
\title{From the Midpoint Convexity to Full Convexity}
\author{}
\date{\today}

\begin{document}

\maketitle

\section*{Abstract}
    We will prove that midpoint convexity implies full convexity under mild conditions.

\section*{Notes}
First, let's define these terms. We say a function $f:(0,1) \to \mathbb R$ is $\lambda$-convex if
$$
f(\lambda x + (1-\lambda)y) \le \lambda f(x) + (1-\lambda) f(y), \quad
\forall x, y \in (0,1).
$$
If $f$ satisfies this condition for $\lambda = 1/2$, it is called midpoint convex.
If $f$ is $\lambda$-convex for any $\lambda \in (0,1)$, then it is fully convex.

To proceed, we define
the set of dyadic numbers (or dyadic rationals), denoted by $\mathbb{D}$, 
as the set of rational numbers of the form:

$$
\mathbb{D} = \cup_{n=0}^\infty \mathbb{D}_n,
$$
where 
$$ \mathbb{D}_n =  \left\{ \frac{k}{2^n} \;\middle|\; k = 0, \ldots, 2^n \right\}.
$$


\begin{proposition}
    If $f:(0,1) \to \mathbb R$ satisfies midpoint convexity and is locally bounded above, then
    \begin{enumerate}
        \item $f$ is locally bounded;
        \item $f$ is $\lambda$-convex for any dyadic number $\lambda$;
        \item $f$ is locally Lipschitz;
        \item $f$ is fully convex.
    \end{enumerate}
\end{proposition}

\begin{proof}
    \begin{enumerate}
        \item \textbf{Local boundedness:} Since 
        $f$ is locally upper bounded, it's enough
        to show that $f$ is locally lower bounded.
        Fix $x_0 \in (0,1)$. We set 
        $$ \delta = \min\{x_0, 1-x_0\}/2.$$
        By assumption, there exists $M \in \mathbb R$ such that
        $$M = \sup_{x \in (x_0 - \delta, x_0 + \delta)} f(x) < +\infty.$$
        For any $x \in (x_0 - \delta, x_0 + \delta)$, we set $y = 2 x_0 - x$.
        Note that $y \in (x_0 - \delta, x_0 + \delta)$ as well.
        By midpoint convexity, we have
        $$ f(x) \ge 2 f(x_0) - f(y) \ge 2 f(x_0) - M.$$
        This shows that $f$ is locally lower bounded on $(0,1)$.
        \item \textbf{Dyadic convexity:} 
        Let $G$ be the set of $\lambda \in [0,1]$ such that $f$ is $\lambda$-convex.
        Immediately from the definition, we have $\mathbb D_0 = \{0,1\} \subset G$.
        Assume that $\mathbb D_n \subset G$ for some $n \ge 0$.
        We will show that $\mathbb D_{n+1} \subset G$.
        Indeed, if $\lambda_1, \lambda_2 \in G$, then by midpoint convexity,
        we have $\frac{\lambda_1 + \lambda_2}{2} \in G$ as well.
        Since any element in $\mathbb D_{n+1}$ is the average of two
        elements in $\mathbb D_n$, we conclude that $\mathbb D_{n+1} \subset G$.
        By induction, we have $\mathbb D_n \subset G$ for all $n \ge 0$,
        and hence $\mathbb D \subset G$.
        \item \textbf{Local Lipschitz continuity:} Fix $x_0 \in (0,1)$.
        We set 
        $$ \delta = \min\{x_0, 1-x_0\}/4.$$
        By local boundedness, there exists $M > 0$ such that
        $$ |f(x)| \le M, \quad \forall x \in (x_0 - 2\delta, x_0 + 2\delta).$$
        Now, we show that $f$ is Lipschitz on $(x_0 - \delta, x_0 + \delta)$
        with a bounded Lipschitz constant. Choose 
        any $x< y \in (x_0 - \delta, x_0 + \delta)$.
        Let $z_n = \max \{y + \delta_n | \delta_n \in \mathbb D_n \cap (0, \delta)\}$.
        Note that 
        $$z_n = y + \delta_n \to y + \delta = z < x_0 + 2\delta \text{ as } n \to \infty.$$
        By dyadic convexity, we have
        $$\frac{f(y) - f(x)}{y - x} \le \frac{f(z_n) - f(y)}{z_n - y} \le \frac{1}{\delta_n} \cdot 2M \to \frac{2M}{\delta}, 
        \text{ as } n \to \infty.$$
        This shows the upper bound of Lipschitz constant on $(x_0 - \delta, x_0 + \delta)$.
        Similarly, we can take $w_n = \min \{x - \delta_n | \delta_n \in \mathbb D_n \cap (0, \delta)\}$. 
        Then, we have
        $$ \frac{f(x) - f(y)}{x - y} \ge \frac{f(x) - f(w_n)}{x - w_n} \ge 
        \frac{1}{\delta_n} \cdot (-2M) \to \frac{-2M}{\delta}, \text{ as } n \to \infty.$$ 
        This shows the lower bound of Lipschitz constant on $(x_0 - \delta, x_0 + \delta)$.
        \item \textbf{Full convexity:} Full convexity follows from continuity and dyadic convexity.  
    \end{enumerate}
\end{proof}

\end{document}